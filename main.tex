\documentclass{article}
\usepackage{graphicx} % Required for inserting images
\usepackage{enumitem}

\title{GitTutorial}
\author{Runpeng Li}
\date{February 2024}

\begin{document}
\maketitle

\section*{Git Basics: A Practical Guide for Beginners}

Git is a distributed version control system that helps you track changes to your codebase. By the end of this tutorial, your students will be well-equipped to use Git for their semester-long projects.

\subsection*{1. What is Git?}
Git allows collaboration, keeps your project history organized, and facilitates seamless teamwork.

\subsection*{2. Installation and Configuration}
\begin{itemize}
    \item Install Git: Download it from the official website or use package managers like Homebrew (for macOS) or Chocolatey (for Windows).
    \item Configure Git: Set up your name and email using the following commands:
    \begin{verbatim}
    git config --global user.name "Your Name"
    git config --global user.email "your@email.com"
    \end{verbatim}
\end{itemize}

\subsection*{3. Creating a Repository}
\begin{itemize}
    \item Initialize a New Repository: Navigate to your project folder and run:
    \begin{verbatim}
    git init
    \end{verbatim}
    \item Cloning an Existing Repository: Use:
    \begin{verbatim}
    git clone <repository_url>
    \end{verbatim}
\end{itemize}

\subsection*{4. Basic Workflow}
\begin{itemize}
    \item Adding Changes: Add files to the staging area before committing them:
    \begin{verbatim}
    git add <file1> <file2>
    \end{verbatim}
    \item Committing Changes: Commit your staged changes with a descriptive message:
    \begin{verbatim}
    git commit -m "Add feature X"
    \end{verbatim}
    \item Pushing and Pulling: Push local commits to the remote repository:
    \begin{verbatim}
    git push origin <branch_name>
    \end{verbatim}
    Pull changes from the remote repository:
    \begin{verbatim}
    git pull origin <branch_name>
    \end{verbatim}
\end{itemize}

\subsection*{5. Branches and Merging}
\begin{itemize}
    \item Creating a Branch: Create a new branch for a feature or bug fix:
    \begin{verbatim}
    git checkout -b feature/my-feature
    \end{verbatim}
    \item Switching Branches: Move between branches:
    \begin{verbatim}
    git checkout <branch_name>
    \end{verbatim}
    \item Merging Branches: Merge changes from one branch into another:
    \begin{verbatim}
    git merge <source_branch>
    \end{verbatim}
    \item Handling Conflicts: Resolve conflicts manually and commit the changes.
\end{itemize}

\subsection*{6. Stashing}
\begin{itemize}
    \item Stash Changes: Temporarily save uncommitted changes:
    \begin{verbatim}
    git stash save "Work in progress"
    \end{verbatim}
    \item Apply Stash: Retrieve stashed changes:
    \begin{verbatim}
    git stash apply
    \end{verbatim}
\end{itemize}

\subsection*{7. Additional Resources}
\begin{itemize}
    \item GitHub: Explore GitHub for collaborative development and hosting repositories.
    \item Practice: Create a sample project, experiment with branches, and practice merging.
\end{itemize}

Remember, Git is a powerful tool, and practice makes perfect. Happy coding!

\end{document}
